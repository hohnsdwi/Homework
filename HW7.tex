\documentclass[12pt]{article}

\usepackage[english]{babel}
\usepackage[utf8x]{inputenc}
\usepackage{amsmath}
\usepackage{graphicx}

\title{Math 480 Homework 7: An Experiment in LaTex and Logic}
\author{Dwight Hohnstein}

\begin{document}
\maketitle Demonstrating the Basics of Logic While Learning LaTex Typesetting 

\begin{abstract}
A brief introduction to propositional calculus.
\end{abstract}

\section{Introduction}
\label{sec:intro}

The project that myself, Jessica Junk and Tanner Missler are working on is the betterment and completion of logic.py modules in the Sage library. In this paper I will go over some basic propositional logic.

\section{STOP, READ NO FURTHER IF...}
\label{sec:stop}

Not mentioned in \ref{sec:intro}, this paper will not have the following:
\begin{enumerate}
    \item Formulas of the form $a^2+b^2=c^2$

    \item Rigorous Mathematical Proofs

    \item Theory that underlies any type of nontrivial mathematical concepts.
\end{enumerate}


\section{Meat and Potatoes: An Introduction to Propositional  Calculus}
\label{mp}
Let P, Q, R represent any proposition. Let $\to$ represent implies. Let $\lnot$ represent not. Consider the following statement: $$P \to \lnot P$$ Interpreted in english, the statement reads "P implies not P." In a more concrete example, let P represent the proposition of having oranges. What the above says that if you have oranges, you can't have not oranges. It's converse is also true, 

$$
\lnot P \to P
$$


\subsection{Truth Tables}

Consider the following statment: 
$$
(P \lor Q) \to R
$$ 
For P, Q, R propositions, and $\lor$ representing the phrase 'or'. A truth table is a table that shows the truth or falsity of a given proposition. As stated above, you can either have P or $\lnot$ P. Having P corresponds to true, and $\lnot$ P false similairily. The truth table goes through all possible truth assignments for P and Q to determine the truth or falsity of R. Table 1 shows just that, toggling all possibilities of Q when P is held true, then all assigments of Q when P is held false. $(P \lor Q)$ only becomes false when both P and Q are both false, and thus we do not have the statement R.  

\begin{table}
\centering
\begin{tabular}{c|c|c|c}
P & Q & $(P \lor Q)$ & R \\\hline
T & T & T & T \\\hline
T & F & T & T \\\hline
F & T & T & T \\\hline
F & F & F & F 
\end{tabular}
\caption{The corresponding truth table for $(P \lor Q) \to R$}
\end{table}


\end{document}